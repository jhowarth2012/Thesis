\chapter{The Top Quark}
\label{Chapter:TheTopQuark}
%\addcontentsline{toc}{chapter}{The Top Quark}

\section{The Standard Model}
Intoduction to the SM of particle physics, explain how it comes about [$SU(3)_{colour}$ x $SU(2)_{isospin}$ x $U(1)_{hypercharge}$] and how gauge invariance leads to most interactions. Explain quarks, generations, left handed and right handed couplings, and V-A interactions (which is obviously important for t->bW and W->lnu.

\section{The Heaviest Particle}
Mention Higgs role in giving fermions mass and that the top quark is the same size as an Itrium atom (or perhaps more relatably, the gold atom) and yet is point like. Explain how it was predicted from SM EW fits that the top mass would be in a range of 140\GeV to 180\GeV, and yet this is still only a constraint based of multiparameter fits, and there remains no reason for the top quark mass to be so much heavier than the other fermions and bosons. (Maybe say something about how it may play a role in EW symmetry breaking.

\section{Strong Production of Top Quark Pairs}
\ttbar\ pair production occurs via two primary channels at hadron colliders, fusion of two gluons or anihilation of two quarks. Production via lepton fusion is possible though no collider has yet been constructed with the necessary centre of mass energy. The leading order production diagrams are shown in figure \ref{fig:ttbar_prod}.

\begin{itemize}
  \item s-channel, t-channel, u-channel production.
  \item gluon gluon dominated at LHC
  \item q qbar from sea quarks
  \item assymetries
\end{itemize}

\section{Top Quark Decay}

\begin{itemize}
  \item $|V_{tb}|$
  \item classification of decay channel
  \item V-A structure
\end{itemize}

\section{Spin Correlations}

\begin{itemize}
  \item Definition of A
  \item Why are Spin Correlations interesting?
  \item Accessing Spin Correlations via angular distributions
  \item Spin analysing power of decay particles (Why we pick the Dilepton Channel)
  \item Inclusive measurements vs mass split measurements
\end{itemize}

%\subsection{Like Helicity Gluons}
%\subsection{Oposite Helicity Gluons and Fermions}

\section{Probing Beyond The Standard Model}




